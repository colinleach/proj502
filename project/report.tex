%%
%\documentclass[twocolumn, tighten]{aastex63}
%\documentclass[twocolumn, twocolappendix, tighten]{aastex631}
\documentclass[twocolumn, twocolappendix, tighten]{aastex631}

\newcommand{\vdag}{(v)^\dagger}
\newcommand\aastex{AAS\TeX}
\newcommand\latex{La\TeX}

\newcommand{\tsub}[1]{\textsubscript{#1}}
\newcommand{\tsup}[1]{\textsuperscript{#1}}
\newcommand{\so}{\qquad \implies \qquad}
\newcommand{\todo}{\color{red}{TODO}\color{black}\hspace{2mm}}


\shorttitle{title}
\shortauthors{Colin Leach}

\graphicspath{{./}{figures/}}

\begin{document}

\title{Term Project \\Galaxy Zoo: Probabilistic Morphology through Bayesian CNNs and Active Learning}

\author[0000-0003-3608-1546]{Colin Leach}

\begin{abstract}

Astronomical survey data has expanded impressively since the era when professional astronomers could keep up with it by themselves. As an early enhancement, Galaxy Zoo used large numbers of amateur volunteers for classification of SDSS results, more recently extended to HST, CANDELS and DECaLS images. To scale further for the Rubin/Euclid era, that approach needs to be supplemented with ML techniques to use the volunteers more efficiently. \citet{walmsley_galaxy_2020} attempts to develop such a hybrid human/ML system. The current term project attempts to reproduce and (perhaps) extend this work.\\

\end{abstract} 

\section{Introduction} \label{sec:intro}

%\paragraph{Literature Background}

The Galaxy Zoo started as an attempt to scale manual classification of SDSS images by recruiting citizen scientists \citep{2008MNRAS.389.1179L}. This succeeded beyond expectations, but is struggling to keep up with new data sources: DES, Rubin, Euclid, etc. Volunteer input is increasingly regarded as a finite and valuable resource, which needs to be used more efficiently \citep{2020IAUS..341...99D}.

Sorting galaxies by color has been done for decades (blue spirals, red ellipticals), though this has been criticized as inaccurate \citep{smethurst_quantifying_2022}. Other approaches include radial brightness curves, looking for central bulges and bars. Attempts to use neural networks to classify morphology go back at least to a Kaggle challenge\footnote{https://www.kaggle.com/c/galaxy-zoo-the-galaxy-challenge} in 2014, won by \citet{2015MNRAS.450.1441D}. The concept of transfer learning, using older surveys to train models for a newer one, was explored by \citet{2019MNRAS.484...93D} and later by W+20, discussed in more detail in \citet{2021arXiv211012735W}. These all focus on visual images (or their redshifted equivalents), but \citet{2021arXiv211104353F} discusses an exchange of techniques with radio astronomy. A broader review of ML in astronomy is given in \citet{2020WDMKD..10.1349F}.

GZ2 \citep{willett_galaxy_2013, 2016MNRAS.461.3663H} is based on SDSS DR7. Later catalogs include Galaxy Zoo: Hubble \citep{2017MNRAS.464.4176W}, CANDELS \citep{2017MNRAS.464.4420S} and DECaLS \citep{walmsley_galaxy_2022}.

\section{Aims}

In \citet{walmsley_galaxy_2020} (hereafter W+20), an attempt is described to develop a human-machine hybrid strategy for galaxy morphology:
\begin{itemize}
	\item Use the large Galaxy Zoo 2 (GZ2) catalog to train a CNN that can classify SDSS images.
	\item Use this model as a starting point to classify new data sources and formats, using only modest amounts of labeling from human volunteers to fine-tune the model.
\end{itemize}


\section{Code} \label{sec:code}

\subsection{Zoobot Code} \label{sec:zcode}

\textbf{Code:} All the Python/Tensorflow code is on Github\footnote{https://github.com/mwalmsley/galaxy-zoo-bayesian-cnn} \citep{walmsley_mwalmsleygalaxy-zoo-bayesian-cnn_2019}, claiming to be an exact copy of that used for W+20.

Perhaps more interesting is the zoobot repo\footnote{https://github.com/mwalmsley/zoobot}, a fork which is still under active development. This extends the project to DECaLS (Dark Energy Camera Legacy Survey) data, as described in \citet{2021MNRAS} (hereafter W+21). It also has much better documentation\footnote{https://zoobot.readthedocs.io/} than the earlier code.

\subsection{Code for Term Project}

Python code and documentation associated with ASTR 502 is available on Github\footnote{https://github.com/colinleach/proj502}. This aims to cover both GZ2, as in W+20, and DECaLS, as in W+21.


\section{Data} \label{sec:data}

\subsection{Catalog Data} 

GZ2 catalogs are available online\footnote{https://data.galaxyzoo.org/} in multiple formats, with 231 columns and nearly 300k rows as described in \citet{willett_galaxy_2013}.

The table used in this work was based on \citet{2016MNRAS.461.3663H}, downloaded as \href{https://zooniverse-data.s3.amazonaws.com/galaxy-zoo-2/zoo2MainSpecz.csv.gz}{a gzipped csv file}.  This is Table 5 in \citet{willett_galaxy_2013} and the column format is described in \href{https://data.galaxyzoo.org/data/gz2/zoo2MainSpecz.txt}{an accompanying file}.

\subsection{Image Data:} 
The GZ team do not make their images library publicly available. However, each $512 \times 512$ image is available from the SDSS cutout service, using the ra/dec coordinates in the catalog table.

Before analysis, the images need to be downsampled to $256 \times 256$ monochrome pixels and stored as uint8.



\section{Computation} \label{sec:comp}

W+20 reports that GZ2 training was carried out on a p2.xlarge EC2 instance with K80 GPU, taking about 8 hours. For DECaLS, the GPU was upgraded to a V100.

Experiments with the GPUs available at the start of this project rapidly proved that 2GB of GPU memory is wholly inadequate for training a CNN. Upgrading to a 6GB GTX 1660 (far from state of the art, but only \$450 and compatible with the existing motherboard and PSU) allowed some progress. However, this still proved limiting for batch size as discussed below.

\todo Colab and AWS


\section{Goals} \label{sec:goals}

My time is less valuable than for faculty or grad students, so goals are open-ended depending on energy, enthusiasm and (hopefully) competence. Roughly:
\begin{enumerate}
	\item Get the published code running on my local machine, using whatever cut-down training set proves viable.
	\item Deploy the code on either AWS or Google.
	\item Extend the model to other data such as Hubble, CANDELS, DECaLS, for which there is already some GZ classification.
	\item Think about newer CNN algorithms. The W+20 paper was submitted in 2019, but software decisions were made well before then and the authors admit it is not the latest technology.
	\item Rewrite using other frameworks, for my education. Most obviously PyTorch, but (unlike most astronomers!) I would also be interested to try Julia with Flux. As a stretch goal, I may try getting it working in F\#/ML.NET, but don't hold your breath waiting for that.
\end{enumerate}

I think we can assume that not all of this will be done before the end of the semester (an understatement).

\section{Algorithms}

\subsection{What are we trying to predict?}

decision tree

\begin{figure}[htb!]
	\plotone{decals_decisions}
	\caption{The GZ decision tree used for DECaLS DR5
		\label{fig:decals_decisions}}
\end{figure}


\subsection{ML model}

EfficientNet B0

\section{Workflow}

In outline, these are the steps required:

\begin{enumerate}
	\item Get the Galaxy Zoo catalog data for each survey of interest.
	\item Get the image files (JPG or PNG), one per galaxy in the classification.
	\item Make a combined catalog, including a path to the image on disk plus the data fields relevant to the model.
	\item Split the galaxies into train, evaluate and test sets. For each, prepare a binary tensor (tfrecord) file containing image and classification data.
	\item Train the model on the train and evaluate sets.
	\item Predict results with the test set and compare with the GZ classification.
\end{enumerate}

The following subsections address each of these in more detail.

\subsection{GZ data}

GZ2 files were downloaded from the Galaxy Zoo website.
There are a total of 243,500 rows in the table. For better consistency, only those marked 'original' in the sample field were used in subsequent analyses, a set of 211,922.

Extensive DECaLS data is available from Zenodo \citep{walmsley_mike_2020_4573248}. For this study the file 'gz\_decals\_volunteers\_5.parquet' was used, a total of 253,286 rows.

For maximum flexibility (and because old habits die hard), all this data was stored in a PostgreSQL database, running locally.

\subsection{Images} \label{images}

The RA and Dec fields in the GZ2 dataset were used to fetch $424 \times 424$ JPG cutouts from the \href{http://skyserver.sdss.org/dr14/SkyServerWS/ImgCutout/getjpeg}{SDSS SkyServer}. Because Zoobot is currently configured to use PNG images, the Python code converted each file with PIL. The PNG files totaled around 33 GB, much more than the corresponding JPG files.

DECaLS DR5 images were downloaded from Zenodo \citep{walmsley_mike_2020_4573248} as 4 ZIP files, unpacked to 272,725 $424 \times 424$ PNG files totaling 83 GB. 

File paths and some metadata was stored in PostgeSQL.

\subsection{Combined catalog}

Having everything in PostgreSQL makes it easy to join the data and image tables and select the desired columns. Each resulting dataset was converted to a pandas DataFrame and saved as a CSV file. This is quick and produces relatively small files (around 35 MB).

Zoobot requires the columns to have the correct names and appear in the correct order. A galaxy identifier is in 'id\_str' and a full path to the PNG is in 'file\_loc', then the remaining columns contain total votes cast for each answer in the GZ decision tree.

\subsection{Tensor shards}

Before training, input data needs to be converted to a tfrecord format that TensorFlow can read quickly. The combined catalog is split into train, evaluate and test sets; for this project a 7:2:1 ratio was used. For each set, image files are read and undergo initial cropping and resizing before combining with the GZ votes and written to binary tfrecord files. This took around 1 hour per survey (i9 processor, local SSD storage) and the output files total about 100 GB.

For debugging, a much smaller GZ2 shard set was also created, with fewer records and low-resolution images. This proved valuable in quickly finding some minor bugs in the current Zoobot repo: apparently it was tested mainly with DECaLS data and there are some typos in the GZ2 code. Accordingly, from this point the project uses my fork of the mwalmsley/zoobot repo. A PR with the corrections will be submitted upstream once everything is working correctly.

\subsection{Training}

As expected, this proved a slow step in the workflow and exposed the limitations of the local (6 GB) GPU. A batch size of 128 was used in the published work. For GZ2, this caused an immediate GPU out-of-memory error. Dropping to batches of 64 was more successful, as in Figure \ref{fig:gz2_train_64}, though this used most of the available GPU memory. Progressing at about 10 min/epoch, the training loss drops smoothly and reached stopping criteria (\todo) after 54 epochs, 8.3 hours. However, the evaluation loss (??) is noisy and suggests rather poor generalization.

\begin{figure}[htb!]
	\plotone{gz2_train_64}
	\caption{
		\label{fig:gz2_train_64}}
\end{figure}

For DECaLS, the batch size needed to be reduced to 32 to fit in GPU memory. Training is slower (about 30 min/epoch) but the results are more encouraging, as shown in Figure \ref{fig:decals_train}. After some initial spikes, the evaluation loss tracks closely with the training loss. This run failed to reach stopping criteria within the epoch limit (40 epochs, nearly 19 hours) but looks good enough to progress with.

It is not immediately clear why the DECaLS run looks better than the GZ2 run. Preparation of data shards uses the same code and no error has yet been found. Other hypotheses include the different batch size and different image size and quality. Batch size is easiest to test, so GZ2 training was repeated with batches of 32 as in Figure \ref{fig:gz2_train_32}. This is not encouraging: training now looks worse without evaluation looking better.

\begin{figure}[htb!]
	\plotone{decals_train}
	\caption{
		\label{fig:decals_train}}
\end{figure}

Images obtained from DECaLS are inherently higher resolution and deeper than those from SDSS used in GZ2 (bigger telescope, newer camera). Training was also carried out on differently sized images: $300 \times 300$ for DECaLS and $256 \times256$ for GZ2 (\todo CHECK).

There are also significant differences in image preparation. W+20 gives little detail about this for GZ2, so the simple method described in section \ref{images} was followed.  In contrast, W+21 describes a more complex process, starting from FITS data files at native telescope resolution. Something equivalent may be possible for SDSS, but as this is not an urgent priority for an ASTR 502 term paper it may be better to focus on the DECaLS survey.

\todo Find galaxies imaged in both surveys to compare by eye

\begin{figure}[htb!]
	\plotone{gz2_train_32}
	\caption{
		\label{fig:gz2_train_32}}
\end{figure}

\section{Predictions}


\bibliography{GZML}{}
\bibliographystyle{aasjournal}


\end{document}

